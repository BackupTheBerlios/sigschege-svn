% Copyright 2004, 2005 by Ingo Hinrichs, Ulf Klaperski
%
% This file is part of Sigschege - Signal Schedule Generator
% 
% #############################################################################
%
% Sigschege is free software; you can redistribute it and/or modify
% it under the terms of the GNU General Public License as published by
% the Free Software Foundation; either version 2, or (at your option)
% any later version.
% 
% Sigschege is distributed in the hope that it will be useful,
% but WITHOUT ANY WARRANTY; without even the implied warranty of
% MERCHANTABILITY or FITNESS FOR A PARTICULAR PURPOSE.  See the
% GNU General Public License for more details.
% 
% You should have received a copy of the GNU General Public License
% along with the Sigschege sources; see the file COPYING.  
%
% #############################################################################
%
% $Id$
%
% Sigschege User Manual
%


\documentclass[11pt]{article}

\begin{document}

\begin{center}
\Huge Sigschege

\vspace{1cm}

\LARGE User Manual

\vspace{3cm}

Created from \verb$Revision$
  
\end{center}

\eject

\section{Introduction}
\label{sec:intro}

Sigschege is an application to create timing diagrams for digital electrical circuits.



\section{Anatomy of a Timing Diagram}
\label{sec:anatomy}



\section{The Python Interface}
\label{sec:python}

\subsection{Running Sigschege}

The \texttt{sigschege} binary is simply an enhanced Python interpreter. The
usual way to create a timing diagram is to write a Python script which uses the
additional functions. We recommend the extension \texttt{.spy} (Sigschege PYthon) for these
scripts. To process a sigschege script just run:

\begin{center}
  \texttt{sigschege my-interface-protocol.spy}
\end{center}


\subsection{Creating a Timing Diagram}

Sigschege implements the \texttt{Sigschege} namespace. To create a timing
diagram just instantiate a \texttt{TimingDiagram} object from this namespace:


\begin{center}
  \texttt{Sigschege.TimingDiagram(<startTime>, <endTime>)}
\end{center}

Both times are floating point values.
This function must be stored in a variable for later access. A real call will
look like:

\begin{center}
  \texttt{td = Sigschege.TimingDiagram(0.0, 50.0)}
\end{center}

\subsection{Implementing Signals}
\label{sec:sig}

Signals can be created with \texttt{createSignal}. It can take the following
arguments:

\begin{description}
\item[label] The label which will be printed in the text area on the left side.
\item[defaultSlope] \emph{optional} The default slope for all events. Default is 0.
\item[before] \emph{optional} Another element of the timing diagram to insert this element before.  
\end{description}

\begin{center}
  \texttt{sig\_clk = td.createSignal(label=''Clock'', defaultSlope=0.5)}
\end{center}

\subsection{Instantiating a Time Scale}
\label{sec:tis}



\subsection{Exporting Timing Diagrams}
\label{sec:exp}



\end{document}

